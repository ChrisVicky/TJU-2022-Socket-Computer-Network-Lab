\chapter{进度总结及项目分工}

\section{本周进度情况}

本周主要完成了对 yacc 和 lex 文件的一定研究与实践,对消息 解析的原理进行了了解。同时,复习了 socket 编程的框架,实现了 socket 编程的实践部分。具体完成情况如表\ref{tab:renwu}所示。

\begin{table}[htbp!]
    \centering
    \begin{tabular}{llll}
    \hline
    \multicolumn{1}{c}{本周任务要求}                                        & 完成 & 没完成 & 备注 \\ \hline 
    1、阅读HTTP/1.1的标准文档RFC2616                      & \checkmark  &     &  无  \\ 
    2、搭建编程环境                                      & \checkmark  &     &  无  \\
    3、熟悉Socket编程方法                                & \checkmark  &     &  无  \\ 
    4、掌握lex和yacc正确解析消息(message)的方法         & \checkmark  &     &  无  \\ 
    5.1、实现简单的echo web server Echo GET, HEAD, POST & \checkmark  &     &  无  \\ 
    5.2、响应没有实现的方法                                 & \checkmark  &     &   无 \\ 
    5.3、响应错误的方法                                   & \checkmark  &     &  无  \\ 
    6、功能测试                                        & \checkmark  &     & 无   \\ \hline
    \end{tabular}
    \caption{本周进度完成表}\label{tab:renwu}
    \end{table}

\section{人员分工}

人员分工如表\ref{tab:fengong}所示。

\begin{longtable}{p{4em} p{14em}}
    \hline
    人员 & \multicolumn{1}{c}{项目分工} \\
    \midrule
        刘锦帆 & 完成消息解析及分发,整体项目的测试与 Debug。同时,完成实验报告相应部分的撰写 \\ \hline        李镇州 & 完成 echo\_server 返回消息的封装与实验报告的框架 \\ 
        \hline
    
      \caption{人员分工表}  \label{tab:fengong}
\end{longtable}


